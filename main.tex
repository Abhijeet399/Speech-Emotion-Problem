% -- Encoding UTF-8 without BOM
% -- XeLaTeX => PDF (BIBER)

\documentclass[espanol]{cv-style}     % Add 'print' as an option into the square bracket to remove colours from this template for printing. 
                                      % Add 'espanol' as an option into the square bracket to change the date format of the Last Updated Text
%\setdefaultlanguage{spanish}
%\sethyphenation[variant=spanish]{}{}  % Add words between the {} to avoid them to be cut 

\usepackage{graphicx}

\begin{document}

\header{Abhijeet  }{Bhattacharya}
%\lastupdated

%----------------------------------------------------------------------------------------
% SIDEBAR SECTION  -- In the aside, each new line forces a line break
%----------------------------------------------------------------------------------------
\begin{aside}
\section{-}
\includegraphics[width=4cm]{IMG_363yh2.jpg}
%
\section{Contact Informations}
    Flat number 71
    Vidya Vihar society
    plot number 48
    Rohini sec-9
    Delhi-110085
    ~
    +91 8527801920
    \href{mailto:bhattacharya399@gmail.com}{bhattacharya399@gmail.com}
    %\href{http://friggeri.net}{http://friggeri.net}
    \href{https://github.com/Abhijeet399}{github.com/Abhijeet399}
%
\section{languages}
    English
    Hindi
    Bengali
  \section{programming}
    Python, C,
    Embedded C,
    Matlab,LATEX
  \section{Skills and Tools}
    Arduino, Raspberry Pi, Deep Learning, Computer Vision,
    Embedded System,Machine Learning,Proteus
%
\end{aside}
%----------------------------------------------------------------------------------------
% RESUMEN SECTION
%----------------------------------------------------------------------------------------
\vspace{0.2cm}
\section{Interests}
  \vspace{-0.2cm}
 \textbf{Research in Deep Learning,R.O.S.(Robotics operating system), Computer Vision, Human face sketching, playing flute,NLP (Natural language processing), Reinforcement Learning.}.
%----------------------------------------------------------------------------------------
% WORK EXPERIENCE SECTION
%----------------------------------------------------------------------------------------
\section{Education}

\begin{entrylist}
  \entry
    {since 2017}
    {B.Tech Student in E.E.E.}
    {Bharati Vidyapeeth's College}
    {\emph{Electrical and Electronics Engineering}}
  \entry
    {2015-2017}
    {Class 11th-12th}
    {D.L.D.A.V. School}
    
  \entry
    {2003–2015}
    {Upto class 10th}
    {Apeejay School Pitampura}
  
\end{entrylist}
\section{Experience}
  \vspace{-0.2cm}
\begin{entrylist}
%------------------------------------------------
\entry
{\scalebox{.8}[1.0]{2018-Present}}
{Research Intern at CSIR-CEERI}
{}
{{Under Dr. Manoj Sharma (Data Scientist)}}\\
%\small{Materias aprobadas y cursando: CBC aprobado con promedio 8; %Álgebra 7; Análisis Matemático 9; Teoría de Juegos 10; Algoritmos %y Estructuras de Datos I 10; Algoritmos y Estructuras de Datos II %(Final Pendiente); Algoritmos y Estructuras de Datos III (Final %Pendiente); Organización del Computador I 9; Organización del %Computador II (Final Pendiente); Sistemas Operativos (Cursando); %Probabilidad y Estadística (Cursando).\\
%Optativas del \textbf{Departamento de Matemática}: Álgebra Lineal %8; Análisis Matemático II 9; Cálculo Avanzado (Final Pendiente)}}
%------------------------------------------------
\entry
{\scalebox{.8}[1.0]{2-3 Oct,2018}}
{Organised WIEHACK BVPIEEE flagship event}
%{EEM N° 7, 9 de Julio, Buenos Aires, Argentina}
{   }
{A 24 hour only women hackathon organised by Women in Engineering,IEEE(WIE) and a few BVPIEEE members}

\entry
{\scalebox{.8}[1.0]{2-3 Feb,2018}}
{Organised INNOVICON a Conference,BVPIEEE}
%{EEM N° 7, 9 de Julio, Buenos Aires, Argentina}
{   }
{A Conference organised by BVPIEEE members on the track of Artificial Intelligence(A.I.)}

\entry
{\scalebox{.8}[1.0]{2018-Present}}
{BVPIEEE Chapter Representative}
%{EEM N° 7, 9 de Julio, Buenos Aires, Argentina}
{   }
{Robotics and Automation society(RAS)}

\entry
{\scalebox{.8}[1.0]{2018}}
{Summer training of Embedded System}
{6th June,2018-6th August,2018}
%{EEM N° 7, 9 de Julio, Buenos Aires, Argentina}
{At Cyborg lab}
%------------------------------------------------
\end{entrylist}
%----------------------------------------------------------------------------------------
% AWARDS SECTION
%----------------------------------------------------------------------------------------
\section{Projects}
  \vspace{-0.2cm}
\begin{entrylist}
%------------------------------------------------
\entry
{\scalebox{.8}[1.0]{2018(Jan)}}
{SNAKE BOT}
{\href{https://www.youtube.com/watch?v=288WwUiQuMc&t=5s}{Video of our project}}
{A snake robot, also known as snake robot, is a biomorphic
hyper-redundant robot that resembles a biological snake.
Snakeroots are most useful in situations where their unique characteristics give them
an advantage over their environment. These environments tend to be long and thin
like pipes or highly cluttered like rubble. Thus snakeroots are currently being
developed to assist search and rescue teams.}
\entry
{\scalebox{.8}[1.0]{2017(Dec)}}
{ATTENDANCE SYSTEM}
{           }
{We as a team are designing a system That
takes attendance on basis of bio-metric system and your name System. This
Project is under Mr. Manoj Sharma (ECE DEPT.) of
our college.}

\entry
{\scalebox{.8}[1.0]{2018(Oct)}}
{DEAF TO MUTE COMMUNICATION MODEL}
{\href{https://drive.google.com/open?id=1_bYviurdXB6xN-S8j15YnuRMcF-p-heS}{Video of our project}}
{We as a team are designing a system that
converts hand gestures to audio voice with the help of Firebase(a Google API) and the values of the potentiometer at each node of the finger on their hand.}

\entry
{\scalebox{.8}[1.0]{2019(Jan)}}
{Image segmentation on Android}
{}
{It is the task given to us by Dr.Manoj Sharma in CEERI as the Winter Internship.}

%------------------------------------------------
\end{entrylist}
  \vspace{-0.2cm}
%----------------------------------------------------------------------------------------
% INTERESTS SECTION
%----------------------------------------------------------------------------------------3
\section{Awards}
  \vspace{-0.2cm}
\textbf{-}  {\href{https://www.youtube.com/watch?v=rrQgRlSMAI8&t=9536s}{WON Rajasthan Digifest Hackathon 5.0 organised by Rajasthan Government.Award Ceremony (2:37:00)}}//

\textbf{-}  {\href{https://twitter.com/DIPRRajasthan/status/1022771192904601601/photo/1}{Awarded with funding of rupees 15Lakh by the CM of Rajasthan Mrs.Vasunra Raje(Award Ceremony)}}//

\textbf{-}  {\href{https://www.facebook.com/pg/hackabit/posts/}{Won the East-India's Largest Hackathon HACK-A-BIT organised by BIT-MESRA}}//

\textbf{-}
 {\href{https://www.coursera.org/account/accomplishments/certificate/E4KL4KCYWS2K}{Certificate for completion of the Deep Learning Course on coursera by Andrew Ng.}}//

%----------------------------------------------------------------------------------------
\end{document}